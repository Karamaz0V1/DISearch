\documentclass{article}
%\usepackage{fullpage}
\usepackage{fancyhdr}
\usepackage[english,francais]{babel}
\usepackage[T1]{fontenc}
\usepackage[utf8]{inputenc}
\usepackage[pdftex]{graphicx}
\usepackage{subfig}

%\renewcommand{\baselinestretch}{2}
\author{Brieuc \textsc{Daniel}, Océane \textsc{Lasserre}, Danchi \textsc{Li}, \\ Florent \textsc{Guiotte}, Frédéric \textsc{Becker}}
\title{Moteur de recherche d'images \\ \Large{Compte rendu de résultats}}
\pagestyle{fancy}

\begin{document}
\maketitle
\tableofcontents

\section{Introduction}

Le but de ce projet est de développer un moteur de recherche dans une base de données. 

Le projet se divise en deux parties, une <<Offline>> qui extrait et indexe des descripteurs sur une base de données
d'images donnée. Une autre dite <<Online>> extrait ce même type de descripteur sur une image requête, et cherche des
correspondances avec la base des descripteurs générée en <<Offline>>. La correspondance se fait à l'aide d'un 
algorithme de vote. Les meilleurs résultats sont alors affichés selon leur pertinence. 

Les images sont indexées via des descripteurs (SURF et SIFT) et choisies par algorithme de vote.

\section{Tests et résultats}
\subsection{Tests avec différents descripteurs}

Sur une même base de données, nous avons testé l'algorithme avec deux descripteurs différents.
\begin{figure}[!ht]%htp]
  \centering
  \subfloat[SIFT]{\label{desc:sift}\includegraphics[width=0.45\textwidth]{img/sift.png}}
  \hspace{0.04\textwidth}
  \subfloat[SURF]{\label{desc:surf}\includegraphics[width=0.45\textwidth]{img/surf.png}}
  \caption{Résultats avec différents descripteurs}
  \label{desc}
\end{figure}

\subsection{Tests sur différentes bases de données}

Avec le même descripteur (SIFT), nous avons testé l'algorithme sur plusieurs bases de données.
\begin{figure}[!ht]%htp]
  \centering
  \subfloat[COREL]{\label{db:corel}\includegraphics[width=0.45\textwidth]{img/corel.png}}
  \hspace{0.04\textwidth}
  \subfloat[NISTER]{\label{db:nister}\includegraphics[width=0.45\textwidth]{img/nister.png}}
  \caption{Résultats avec différentes bases de données}
  \label{desc}
\end{figure}

\section{Conclusion}

Sur nos données de test, le descripteur le plus efficace est SIFT. 

\end{document}

%\begin{figure}[!ht]%htp]
%  \centering
%  \subfloat[Masques]{\label{mask:origine}\includegraphics[width=0.7\textwidth]{img/mer_app_mask.png}}
%  \hspace{0.030\textwidth}
%  \subfloat[Seam carving avec masque]{\label{mask:sc}\includegraphics[width=1.0\textwidth]{img/mer_mask_reduced.png}}
%  \caption{Application des masques et résultat final}
%  \label{mer:init}
%\end{figure}

