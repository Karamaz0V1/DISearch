\documentclass{article}
%\usepackage{fullpage}
\usepackage{fancyhdr}
\usepackage[english,francais]{babel}
\usepackage[T1]{fontenc}
\usepackage[utf8]{inputenc}
\usepackage[pdftex]{graphicx}
\usepackage{subfig}

%\renewcommand{\baselinestretch}{2}
\author{Brieuc \textsc{Daniel}, Océane \textsc{Lasserre}, Danchi \textsc{Li}, \\ Florent \textsc{Guiotte}, Frédéric \textsc{Becker}}
\title{Moteur de recherche d'images \\ \Large{Compte rendu de résultats}}
\pagestyle{fancy}

\begin{document}
\maketitle
\tableofcontents

\section{Introduction}
Le but de ce projet est de développer un moteur de recherche dans une base de données. 

Les images sont indexées via des descripteurs (SURF et SIFT) et choisies par algorithme de vote.

\section{Tests et résultats}
\subsection{Tests avec différents descripteurs}
Sur une même base de données, nous avons testé l'algorithme avec deux descripteurs différents.

\subsection{Tests sur différentes bases de données}
Avec le même descripteur, nous avons testé l'algorithme sur plusieurs bases de données.

\section{Conclusion}
Sur nos données de test, le descripteur le plus efficace est SIFT. 

\end{document}

%\begin{figure}[!ht]%htp]
%  \centering
%  \subfloat[Masques]{\label{mask:origine}\includegraphics[width=0.7\textwidth]{img/mer_app_mask.png}}
%  \hspace{0.030\textwidth}
%  \subfloat[Seam carving avec masque]{\label{mask:sc}\includegraphics[width=1.0\textwidth]{img/mer_mask_reduced.png}}
%  \caption{Application des masques et résultat final}
%  \label{mer:init}
%\end{figure}

